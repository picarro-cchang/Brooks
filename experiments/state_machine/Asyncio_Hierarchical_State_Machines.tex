\documentclass[12pt]{article}
\usepackage{amsmath}
\usepackage{amssymb}
\usepackage{amsthm}
\usepackage{graphicx}
\usepackage{hyperref}
\usepackage{xcolor}
\usepackage{minted}
\usemintedstyle{vs}
\usepackage[latin1]{inputenc}
\usepackage[left=2.00cm, right=2.00cm, top=2.00cm, bottom=2.00cm]{geometry}
\newtheorem{theorem}{Theorem}[section]
\newtheorem*{problem}{Problem}
\newtheorem*{proposition}{Proposition}
\newtheorem*{corollary}{Corollary}
\newtheorem*{lemma}{Lemma}
\definecolor{light-gray}{gray}{0.95}
\title{\textbf{Running Hierarchical State Machines in Python with Asyncio}}
\author{}
\date{}
\begin{document}
\maketitle
This document describes the \texttt{async\_hsm} package for running Hierarchical State Machines (HSM) in Python using the single-threaded asyncio paradigm. The core algorithm is based on work by Miro Samek described in the book ``Practical UML Statecharts in C/C++'' and the code was forked from the \texttt{farc} project by Dean Hall.

Like a normal finite state machine (FSM), an HSM consists of states which are connected by transitions which are triggered by \emph{events}. The hierarchical aspect of HSMs allows states to be nested within one other in parent-child relationships. The machine is in only one state at a time. Each state may have an entry action and an exit action specified, which are performed when the state is entered or exited. Each state recognizes a set of events, and the arrival of such an event will cause an action to take place. The event may or may not specify a transition to a target state. If an event is not recognized by a state, the enclosing (ancestor) states in the hierarchy are examined until one is found that does handle the event. If the event does not specify a state transition, the action is performed and the machine remains in the \emph{original} (inner) state, even though the handler is defined in the outer state. On the other hand, if the event \emph{does} cause a transition to a target state, the action is performed
and all the exit actions associated with going from the original state to the outer state which handles the event are obeyed before making the transition to the target state. If all the ancestors of a state do not handle an event, it is silently handled by an implicit top state which is defined as the common ancestor of all the states. This causes no state transition, and so the event is effectively ignored. 

Performing a transition between two states in an HSM involves exiting states up to the last common ancestor (LCA) followed by entering states to the target state. All the exit and entry actions along the path are carried out. Note that we distinguish between remaining in the same state and transitioning from a state to itself. When transitioning from a state to itself, the exit action for the state is performed followed by the entry action. When remaining in a state, neither entry nor exit actions are performed. Following a transition to the target state, any initialization action defined for that state is performed, which will result in further state transitions.

When an HSM handles an event, the transitions and actions that it causes run to completion. In other words, any events that occur while the original event is being handled are just placed on an event queue,  whether they arise from external sources or are generated within the actions performed during the processing. The next event is not fetched from the event queue until after processing of the first event is complete.

\begin{figure}[h]
	\centering
	\includegraphics[width=0.7\linewidth]{SM_of_Example1}
	\caption[Example of state transitions in an HSM]{Example state transitions in an HSM}
	\label{fig:smofexample1}
\end{figure}

Referring to Figure \ref{fig:smofexample1}, let us consider the behavior of the HSM in response to several events.
\begin{itemize}
	\item Since the initial state is defined to be \texttt{state1}, \texttt{ENTRY action 1} will be executed as the state is entered. Since \texttt{INIT action 1} is defined, this will be performed next, followed by a transition to \texttt{state2}, which causes execution of \texttt{ENTRY action 2}.
	
	\item Suppose that \texttt{EVENT1} is received. Since this is handled by \texttt{state2}, the action \texttt{EVENT1 action 2} is performed followed by a transition to \texttt{state1} which performs \texttt{EXIT action 2}. Note that we do \emph{not} exit \texttt{state1}. Since \texttt{INIT action 1} is defined, this will be performed next, followed by a transition to \texttt{state2}, which causes execution of \texttt{ENTRY action 2}.
	
	\item Next, suppose that \texttt{EVENT2} is received. Since this is handled by \texttt{state2}, the action \texttt{EVENT2 action 2} is performed followed by a transition to \texttt{state3} which performs \texttt{EXIT action 2} followed by \texttt{ENTRY action 3}.
	
	\item Next, suppose that \texttt{EVENT1} is received. Since this is not handled by \texttt{state3}, we examine the parent \texttt{state1} which does handle it. This involves performing \texttt{EVENT1 action 1} but \emph{no} state transition, which leaves the machine in \texttt{state3}.
	
	\item Finally, suppose that \texttt{EVENT2} is received. Since this is not handled by \texttt{state3}, we examine the parent \texttt{state1} which does handle it. The action \texttt{EVENT 2 action 1} is performed, which is followed by a transition, so we exit \texttt{state3}, performing \texttt{EXIT action 3} to get up to \texttt{state1}. We then perform the transition from \texttt{state1} to itself. As discussed previously, this causes \texttt{EXIT action 1} followed by \texttt{ENTER action 1} to be performed. Since \texttt{INIT action 1} is defined, this will be performed next, followed by a transition to \texttt{state2}, which causes execution of \texttt{ENTRY action 2}.
		
\end{itemize}

The event handling portion of an HSM is coded in the class \texttt{Hsm}. Instances of this class have a method called \texttt{dispatch} which takes an \texttt{Event} and performs all the actions and state transitions caused by that event before returning. The class \texttt{Ahsm} (an Augmented Hierarchical State Machine) is a subclass of \texttt{Hsm} which adds an event queue together with methods to post events to the queue using the FIFO or the LIFO discipline. In order to run a single HSM, it would be possible to write a task which fetches from the event queue and calls the \texttt{dispatch} method to process that event to completion before looping to fetch the next event. In the \texttt{async\_hsm} package, a separate \texttt{Framework} class is provided which allows a collection of inter-communicating \texttt{Ahsm} instances to be run concurrently. The operation of the \texttt{Framework} will be described in more detail later.

In normal use, an HSM is specified by subclassing \texttt{Ahsm}. Its operation is defined by writing ``state methods,'' one for each state of the machine. State transitions take place in response to namedtuples of type \texttt{Event}. Each such \texttt{event} has two parts, the first \texttt{event.signal} indicates the type of the event, while the second \texttt{event.value} can be any payload associated with the event. The type of an event is a \texttt{Signal}, which effectively acts as an enumeration. In order to create a signal named \texttt{SIGUSER}, the name is registered with the class by calling \texttt{Signal.register("SIGUSER")}. After performing this registration, we may use the notation \texttt{Signal.SIGUSER} and construct an event such as \texttt{Event(Signal.SIGUSER, payload)} which has \texttt{event.signal = Signal.SIGUSER} and \texttt{event.value = payload}.

The following code listing shows how the HSM in Figure 1 may be encoded as methods a class:

\begin{minted}
[
baselinestretch=1.0,
bgcolor=light-gray,
fontsize=\footnotesize,
linenos
]	
{python}
from async_hsm import Ahsm, Event, Signal, state

class HsmExample1(Ahsm):
    @state
    def _initial(self, event):
        Signal.register("E1")
        Signal.register("E2")
        return self.tran(self.state1)

    @state
    def state1(self, e):
        sig = e.signal
        if sig == Signal.ENTRY:
            print("ENTRY action 1")
            return self.handled(e)
        elif sig == Signal.EXIT:
            print("EXIT action 1")
            return self.handled(e)
        elif sig == Signal.INIT:
            print("INIT action 1")
            return self.tran(self.state2)
        elif sig == Signal.E1:
            print("Event 1 action 1")
            return self.handled(event)
        elif sig == Signal.E2:
            print("Event 2 action 1")
            return self.tran(self.state1)
        return self.super(self.top)

    @state
    def state2(self, e):
        sig = e.signal
        if sig == Signal.ENTRY:
            print("ENTRY action 2")
            return self.handled(e)
        elif sig == Signal.EXIT:
            print("EXIT action 2")
            return self.handled(e)
        elif sig == Signal.E1:
            print("Event 1 action 2")
            return self.tran(self.state1)
        elif sig == Signal.E2:
            print("Event 2 action 2")
            return self.tran(self.state3)
        return self.super(self.state1)


    @state
    def state3(self, e):
        sig = e.signal
        if sig == Signal.ENTRY:
            print("ENTRY action 3")
            return self.handled(e)
        elif sig == Signal.EXIT:
            print("EXIT action 3")
            return self.handled(e)
        return self.super(self.state1)
\end{minted}
Each state method is decorated using \texttt{@state}. A state function is invoked with an argument \texttt{e} which is the event that it needs to handle. As mentioned previously, \texttt{e.signal} is a signal defining the type of the event and \texttt{e.value} is the payload. Every state function must return one of the following, depending on the type of the signal

\begin{itemize}
	\item \texttt{self.handled(e)}. This indicates that the event has been handled and should not cause a state transition. Events of type \texttt{Signal.ENTRY} and \texttt{Signal.EXIT} should always return in this way if they are handled.
	
	\item \texttt{self.tran(next\_state)}. This indicates that the machine should transition to \texttt{next\_state} (which is a state method) when an event of this type occurs. An event of type \texttt{Signal.INIT} should return with a transition to a substate of the current state if it is handled.

	\item \texttt{self.super(parent\_state)}. This should be the default return value. The method gets here if the event is not explicitly handled within this state. Note that this default return value informs the code of the identity of the parent of this state.

\end{itemize}


\begin{minted}
[
baselinestretch=1.0,
bgcolor=light-gray,
fontsize=\footnotesize,
linenos
]	
{python}
if __name__ == "__main__":
    s1 = HsmExample1()
    s1.init()
    while True:
        sig_name = input('\tEvent --> ')
        try:
            sig = getattr(Signal, sig_name)
        except LookupError:
            print("\nInvalid signal name", end="")
            continue
        event = Event(sig, None)
        s1.dispatch(event)
\end{minted}

\begin{minted}
[
baselinestretch=1.0,
bgcolor=light-gray,
fontsize=\footnotesize,
linenos
]
{text}
ENTRY action 1
INIT action 1
ENTRY action 2
        Event --> E1
Event 1 action 2
EXIT action 2
INIT action 1
ENTRY action 2
        Event --> E2
Event 2 action 2
EXIT action 2
ENTRY action 3
        Event --> E1
Event 1 action 1
        Event --> E2
Event 2 action 1
EXIT action 3
EXIT action 1
ENTRY action 1
INIT action 1
ENTRY action 2
        Event -->
\end{minted}

\end{document}